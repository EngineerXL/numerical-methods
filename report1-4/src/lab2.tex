\section{Численные методы решения нелинейных уравнений}

\subsection{Решение нелинейных уравнений}

\subsubsection{Постановка задачи}
Реализовать методы простой итерации и Ньютона решения нелинейных уравнений в виде программ, задавая в качестве входных данных точность вычислений. С использованием разработанного программного обеспечения найти положительный корень нелинейного уравнения (начальное приближение определить графически). Проанализировать зависимость погрешности вычислений от количества итераций.

\subsubsection{Консоль}
\begin{alltt}
$ make
g++ -g -pedantic -std=c++17 -Wall -Wextra -Werror main.cpp -o solution
$ cat tests/2.in
0.5 1 0.000001
$ ./solution < tests/2.in
x_0 = 0.774356940
Решение методом простой итерации получено за 9 итераций
x_0 = 0.774356593
Решение методом Ньютона получена за 5 итераций
$ cat tests/3.in
0.5 1 0.000000001
$ ./solution < tests/3.in
x_0 = 0.774356594
Решение методом простой итерации получено за 14 итераций
x_0 = 0.774356593
Решение методом Ньютона получена за 5 итераций
\end{alltt}
\pagebreak

\subsubsection{Исходный код}
\lstinputlisting{../lab2_1/solver.hpp}
\pagebreak

\subsection{Решение нелинейных систем уравнений}

\subsubsection{Постановка задачи}
Реализовать методы простой итерации и Ньютона решения систем нелинейных уравнений в виде программного кода, задавая в качестве входных данных точность вычислений. С использованием разработанного программного обеспечения решить систему нелинейных уравнений (при наличии нескольких решений найти то из них, в котором значения неизвестных являются положительными); начальное приближение определить графически. Проанализировать зависимость погрешности вычислений от количества итераций.

\subsubsection{Консоль}
\begin{alltt}
$ make
g++ -g -pedantic -std=c++17 -Wall -Wextra -Werror main.cpp -o solution
$ cat tests/2.in
0 1
1 2
0.000001
$ ./solution < tests/2.in
x_0 = 0.832187878, y0 = 1.739406197
Решение методом простой итерации получено за 77 итераций
x_0 = 0.832187922, y0 = 1.739406179
Решение методом Ньютона получена за 4 итераций
$ cat tests/3.in
0 1
1 2
0.000000001
$ ./solution < tests/3.in
x_0 = 0.832187922, y0 = 1.739406179
Решение методом простой итерации получено за 111 итераций
x_0 = 0.832187922, y0 = 1.739406179
Решение методом Ньютона получена за 4 итераций
\end{alltt}
\pagebreak

\subsubsection{Исходный код}
\lstinputlisting{../lab2_2/system_solver.hpp}
\pagebreak
