\section{Вычислительные методы линейной алгебры}

\subsection{LU-разложение матриц. Метод Гаусса}

\subsubsection{Постановка задачи}
Реализовать алгоритм LU-разложения матриц (с выбором главного элемента) в виде программы. Используя разработанное программное обеспечение, решить систему линейных алгебраических уравнений (СЛАУ). Для матрицы СЛАУ вычислить определитель и обратную матрицу.

\subsubsection{Консоль}
\begin{alltt}
$ make
g++ -g -pedantic -std=c++17 -Wall -Wextra -Werror main.cpp -o solution
$ cat tests/3.in
4
-7 3 -4 7
8 -1 -7 6
9 9 3 -6
-7 -9 -8 -5
-126 29 27 34
$ ./solution < tests/3.in
Решение системы:
x1 = 8.000000
x2 = -9.000000
x3 = 2.000000
x4 = -5.000000
Определитель матрицы: 16500.000000
Обратная матрица:
-0.054545, 0.054545, 0.006061, -0.018182
0.086000, -0.016000, 0.082667, 0.002000
-0.059818, -0.050182, -0.058909, -0.073273
0.017273, 0.032727, -0.063030, -0.060909
\end{alltt}
\pagebreak

\subsubsection{Исходный код}
\lstinputlisting{../lab1_1/lu.hpp}
\pagebreak

\subsection{Метод прогонки}

\subsubsection{Постановка задачи}
Реализовать метод прогонки в виде программы, задавая в качестве входных данных ненулевые элементы матрицы системы и вектор правых частей. Используя разработанное программное обеспечение, решить СЛАУ с трехдиагональной матрицей.

\subsubsection{Консоль}
\begin{alltt}
$ cat tests/3.in
5
-7 -6
6 12 0
-3 5 0
-9 21 8
-5 -6
-75 126 13 -40 -24
$ make
g++ -g -pedantic -std=c++17 -Wall -Wextra -Werror main.cpp -o solution
$ cat tests/3.in
5
-7 -6
6 12 0
-3 5 0
-9 21 8
-5 -6
-75 126 13 -40 -24
$ ./solution < tests/3.in
Решение системы:
x1 = 3.000000
x2 = 9.000000
x3 = 8.000000
x4 = 0.000000
x5 = 4.000000
\end{alltt}
\pagebreak

\subsubsection{Исходный код}
\lstinputlisting{../lab1_2/tridiag.hpp}
\pagebreak

\subsection{Итерационные методы решения СЛАУ}

\subsubsection{Постановка задачи}
Реализовать метод простых итераций и метод Зейделя в виде программ, задавая в качестве входных данных матрицу системы, вектор правых частей и точность вычислений. Используя разработанное программное обеспечение, решить СЛАУ. Проанализировать количество итераций, необходимое для достижения заданной точности.

\subsubsection{Консоль}
\begin{alltt}
$ make
g++ -g -pedantic -std=c++17 -Wall -Wextra -Werror main.cpp -o solution
$ cat tests/3.in
4 0.000000001
28 9 -3 -7
-5 21 -5 -3
-8 1 -16 5
0 -2 5 8
-159 63 -45 24
$ ./solution < tests/3.in
Метод простых итераций
Решени получено за 53 итераций
Решение системы:
x1 = -6.000000
x2 = 3.000000
x3 = 6.000000
x4 = 0.000000
Метод Зейделя
Решени получено за 20 итераций
Решение системы:
x1 = -6.000000
x2 = 3.000000
x3 = 6.000000
x4 = 0.000000
\end{alltt}
\pagebreak

\subsubsection{Исходный код}
\lstinputlisting{../lab1_3/iteration.hpp}
\pagebreak

\subsection{Метод вращений}

\subsubsection{Постановка задачи}
Реализовать метод вращений в виде программы, задавая в качестве входных данных матрицу и точность вычислений. Используя разработанное программное обеспечение, найти собственные значения и собственные векторы симметрических матриц. Проанализировать зависимость погрешности вычислений от числа итераций.

\subsubsection{Консоль}
\begin{alltt}
$ make
g++ -g -pedantic -std=c++17 -Wall -Wextra -Werror main.cpp -o solution
$ cat tests/3.in
3 0.000001
-7 -6 8
-6 3 -7
8 -7 4
$ ./solution < tests/3.in
Собственные значения:
l_1 = -11.607818
l_2 = 15.020412
l_3 = -3.412593
Собственные векторы:
0.905671, -0.412378, 0.098514
0.190483, 0.603339, 0.774402
-0.378784, -0.682588, 0.624977
Решение получено за 7 итераций
\end{alltt}
\pagebreak

\subsubsection{Исходный код}
\lstinputlisting{../lab1_4/rotation.hpp}
\pagebreak

\subsection{QR алгоритм}

\subsubsection{Постановка задачи}
Реализовать алгоритм QR – разложения матриц в виде программы. На его основе разработать программу, реализующую QR – алгоритм решения полной проблемы собственных значений произвольных матриц, задавая в качестве входных данных матрицу и точность вычислений. С использованием разработанного программного обеспечения найти собственные значения матрицы.

\subsubsection{Консоль}
\begin{alltt}
$ make
g++ -g -pedantic -std=c++17 -Wall -Wextra -Werror main.cpp -o solution
$ cat tests/3.in
3 0.000001
-1 4 -4
2 -5 0
-8 -2 0
$ ./solution < tests/3.in
Решение получено за 32 итераций
Собственные значения:
l_1 = -7.547969
l_2 = 5.664787
l_3 = -4.116817
\end{alltt}
\pagebreak

\subsubsection{Исходный код}
\lstinputlisting{../lab1_5/qr_algo.hpp}
\pagebreak
