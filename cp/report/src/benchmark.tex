\section{Тест производительности}
В тесте сравниваются перемножение в столбик и с использованием БПФ для чисел с разным числом разрядов.

\begin{longtable}{|p{4.5cm}|p{3cm}|p{3cm}|}
    \hline
    Количество разрядов&БПФ, мс&В столбик, мс\\
    \hline
    \rowcolor{lightgray}
    \multicolumn{3}{|c|} {}\\
    \hline
    $100$&0.973&0.213\\
    \hline
    $200$&1.324&0.362\\
    \hline
    $500$&1.924&1.862\\
    \hline
    $1000$&3.057&7.566\\
    \hline
    $2000$&4.555&28.194\\
    \hline
    $5000$&18.859&178.206\\
    \hline
    $10^4$&37.957&677.440\\
    \hline
    $2\cdot10^4$&80.613&2845.117\\
    \hline
    $5\cdot10^4$&168.944&17920.872\\
    \hline
    $10^5$&351.770&-\\
    \hline
    $2\cdot10^5$&746.176&-\\
    \hline
    $5\cdot10^5$&1617.317&-\\
    \hline
    $10^6$&3445.852&-\\
    \hline
\end{longtable}

<<->> в таблице означает, что замеры времени не производились.

Видно, что при увеличении числа разрядов вдвое, время работы перемножения в столбик возрастает в четыре раза, а время работы перемножения с использованием БПФ примерно в два раза.

\pagebreak
