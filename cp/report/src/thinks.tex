\section{Выводы}
В ходе выполнения курсового проекта я изучил алгоритм быстрого вычисления дискретного преобразования Фурье --- быстрое преобразование Фурье.

Основной сложностью было вывести и понять все формулы, которые используются для БПФ. Детали в статьях опускаются, поэтому не всегда очевиден тот или иной переход.

Интересно было узнать, как вычислять БПФ без дополнительной памяти --- с помощью так называемого преобразования бабочки, иллюстрации к которому довольно красивые.

В работе перемножаются два числа в столбик и с помощью БПФ. В сравнении видно, насколько медленно перемножение в столбик для сравнительно небольшого числа разрядов. При написании своей длинной арифметики важно использовать БПФ для лучшей производительности.

В течении прошлых лет пытался изучить этот алгоритм, но каждый раз не хватало нескольких деталей для полного понимания. Перечитав больше статей, нашёл необходимые детали. Выбрав эту тему даже появилась идея выложить работу в форме статьи, если другие люди поймут алгоритм, прочитав работу.

\pagebreak
