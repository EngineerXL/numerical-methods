\graphicspath{{png/}}

\section{Описание}

\subsection*{Наивное перемножение}

Обозначим $n$ и $m$ --- количество разрядов в числе $a$ и $b$ соответственно. Для перемножения двух чисел в столбик требуется сопоставлять каждому разряду числа $a$ каждый разряд числа $b$. То есть временная сложность такого алгоритма $O(n \cdot m)$.

\subsection*{Дискретное преобразование Фурье (ДПФ)}

Пусть $A(x)$ --- многочлен $n - 1$ степени:
$$A(x) = a_0 \cdot + a_1 \cdot x + a_2 \cdot x^2 + a_3 \cdot x^3 + \dots + a_{n - 2} \cdot a ^ {n - 2} + a_{n - 1} \cdot a ^ {n - 1}$$
Будем считать, что $n$ является степенью двойки. Если это не так, то мы можем дополнить исходный многочлен недостающими членами с нулевыми коэффициентами.

Обозначим $\omega_{n, k}$ --- $k$-й комплексный корень степени $n$ из единицы. Так как $1 = e ^ {2 \pi i}$, то решая уравнение $z ^ n = 1 = e ^ {2 \pi i}$ получим $z_k = e ^ {k \cdot {{2 \pi i} \over {n}}} = \omega_{n, k}$, где $k \in \{0, 1, 2, \dots n - 1 \}$. Обозначим $\omega_n = \omega_{n, 1}$. По свойству степени $\omega_{n, k} = \omega_n ^ k$.

Назовём дискретным преобразованием Фурье многочлена $A(x)$ вектор из $n$ чисел:
$$ DFT(a_0, a_1, a_2, \dots a_{n - 1}) = ( A(\omega_{n} ^ {0}), A(\omega_{n} ^ {1}), A(\omega_{n} ^ {2}), \dots A(\omega_{n} ^ {n - 1}) ) ^ T = (y_0, y_1, y_2, \dots y_{n - 1}) ^ T $$
Заметим, что для $\omega_{n} ^ {k}$ значение многочлена равно:
\begin{equation*}
    \begin{split}
        A(\omega_{n} ^ {k}) & = a_0 \cdot (\omega_{n} ^ {k}) ^ {0} + a_1 \cdot (\omega_{n} ^ {k}) ^ {1} + a_2 \cdot (\omega_{n} ^ {k}) ^ {2} + a_3 \cdot (\omega_{n} ^ {k}) ^ {3} + \dots + a_{n - 1} \cdot (\omega_{n} ^ {k}) ^ {n - 1} = \\
        & = a_0 \cdot \omega_{n} ^ {0} + a_1 \cdot \omega_{n} ^ {k} + a_2 \cdot \omega_{n} ^ {2 \cdot k} + a_3 \cdot \omega_{n} ^ {3 \cdot k} + \dots + a_{n - 1} \cdot \omega_{n} ^ {(n - 1) \cdot k}
    \end{split}
\end{equation*}

Тогда преобразование можно записать в матричном виде:
$$ DFT(a_0, a_1, a_2, \dots a_{n - 1}) =
\begin{pmatrix}
y_0 \\
y_1 \\
y_2 \\
\dots \\
y_{n - 1} \\
\end{pmatrix}
= W \cdot
\begin{pmatrix}
a_0 \\
a_1 \\
a_2 \\
\dots \\
a_{n - 1} \\
\end{pmatrix}
$$
Здесь $W$ --- матрица Вандермонда:
$$ W =
\begin{pmatrix}
{\omega}_{n} ^ {0} & {\omega}_{n} ^ {0} & {\omega}_{n} ^ {0} & {\omega}_{n} ^ {0} & \dots & {\omega}_{n} ^ {0} \\
{\omega}_{n} ^ {0} & {\omega}_{n} ^ {1} & {\omega}_{n} ^ {2} & {\omega}_{n} ^ {3} & \dots & {\omega}_{n} ^ {(n - 1)} \\
{\omega}_{n} ^ {0} & {\omega}_{n} ^ {2} & {\omega}_{n} ^ {4} & {\omega}_{n} ^ {6} & \dots & {\omega}_{n} ^ {2 \cdot (n - 1)} \\
{\omega}_{n} ^ {0} & {\omega}_{n} ^ {3} & {\omega}_{n} ^ {6} & {\omega}_{n} ^ {9} & \dots & {\omega}_{n} ^ {3 \cdot (n - 1)} \\
\dots & \dots & \dots & \dots & \dots & \dots \\
{\omega}_{n} ^ {0} & {\omega}_{n} ^ {(n - 1)} & {\omega}_{n} ^ {2 \cdot (n - 1)} & {\omega}_{n} ^ {3 \cdot (n - 1)} & \dots & {\omega}_{n} ^ {(n - 1) ^ 2}
\end{pmatrix}
$$
Умножим обе части слева на $W^{-1}$, получим:
$$
W^{-1} \cdot
\begin{pmatrix}
y_0 \\
y_1 \\
y_2 \\
\dots \\
y_{n - 1} \\
\end{pmatrix}
=
\begin{pmatrix}
a_0 \\
a_1 \\
a_2 \\
\dots \\
a_{n - 1} \\
\end{pmatrix}
$$
$W^{-1}$ имеет следующий вид:
$$ W^{-1} = {1 \over n} \cdot
\begin{pmatrix}
{\omega}_{n} ^ {0} & {\omega}_{n} ^ {0} & {\omega}_{n} ^ {0} & {\omega}_{n} ^ {0} & \dots & {\omega}_{n} ^ {0} \\
{\omega}_{n} ^ {0} & {\omega}_{n} ^ {-1} & {\omega}_{n} ^ {-2} & {\omega}_{n} ^ {-3} & \dots & {\omega}_{n} ^ {-(n - 1)} \\
{\omega}_{n} ^ {0} & {\omega}_{n} ^ {-2} & {\omega}_{n} ^ {-4} & {\omega}_{n} ^ {-6} & \dots & {\omega}_{n} ^ {-2 \cdot (n - 1)} \\
{\omega}_{n} ^ {0} & {\omega}_{n} ^ {-3} & {\omega}_{n} ^ {-6} & {\omega}_{n} ^ {-9} & \dots & {\omega}_{n} ^ {-3 \cdot (n - 1)} \\
\dots & \dots & \dots & \dots & \dots & \dots \\
{\omega}_{n} ^ {0} & {\omega}_{n} ^ {-(n - 1)} & {\omega}_{n} ^ {-2 \cdot (n - 1)} & {\omega}_{n} ^ {-3 \cdot (n - 1)} & \dots & {\omega}_{n} ^ {-(n - 1) ^ 2}
\end{pmatrix}
$$
Определим обратное преобразование Фурье для вектора из $n$ чисел:
$$
InverseDFT( y_0, y_1, y_2, \dots y_{n - 1} ) = (a_0, a_1, a_2, \dots a_{n - 1}) ^ T
$$
Или в матричном виде:
$$
InverseDFT( y_0, y_1, y_2, \dots y_{n - 1} ) =
\begin{pmatrix}
a_0 \\
a_1 \\
a_2 \\
\dots \\
a_{n - 1} \\
\end{pmatrix}
=
W^{-1} \cdot
\begin{pmatrix}
y_0 \\
y_1 \\
y_2 \\
\dots \\
y_{n - 1} \\
\end{pmatrix}
$$
Очевидно, что $InverseDFT(DFT(A)) = A$.

Вычисление $DFT(A)$ простым перемножение матрицы на вектор требует $O(n ^ 2)$ операций, однако можно добиться существенного ускорения.

\pagebreak

\subsection*{Быстрое преобразование Фурье (БПФ)}
Разобьём исходный многочлен $A(x)$ на два многочлена $A_0(x)$ и $A_1(x)$, содержащие чётные и нечётные коэффициенты соответственно:
$$A_0(x) = a_0 + a_2 \cdot x + a_4 \cdot x ^ 2 + \dots + a_{n - 2} \cdot x ^ {{n \over 2} - 1}$$
$$A_1(x) = a_1 + a_3 \cdot x + a_5 \cdot x ^ 2 + \dots + a_{n - 1} \cdot x ^ {{n \over 2} - 1}$$
Тогда мы можем вычислить значение исходного многочлена следующим образом:
$$ A(x) = A_0(x ^ 2) + x \cdot A_1(x ^ 2) $$
Вычислим дискретное преобразование Фурье для многочленов $A_0(x)$ и $A_1(x)$:
$$ DFT(A_0) = (y^0_0, y^0_1, y^0_2, \dots y^0_{ {n \over 2} - 1}) ^ T $$
$$ DFT(A_1) = (y^1_0, y^1_1, y^1_2, \dots y^1_{ {n \over 2} - 1}) ^ T $$
Здесь $y^0$ и $y^1$ --- значения ДПФ для $A_0(x)$ и $A_1(x)$, а не показатели степени при $y$.

Теперь вычислим $DFT(A)$, используя $DFT(A_0)$ и $DFT(A_1)$, используя следующие три важных свойства:
$${\omega^2_{n, k}} = (e ^ {k \cdot {{2 \pi i} \over {n}}}) ^ 2 = e ^ {2 \cdot k \cdot {{2 \pi i} \over {n}}} = e ^ {k \cdot {{2 \pi i} \over {n \over 2}}} = {{\omega}_{{n \over 2}, k}}$$
$${{\omega}_{n, {k + {n \over 2}}}} = e ^ {{(k + {n \over 2})} \cdot {{2 \pi i} \over {n}}} = e ^ {k \cdot {{2 \pi i} \over {n}}} \cdot e ^ {{n \over 2} \cdot {{2 \pi i} \over {n}}} = e ^ {k \cdot {{2 \pi i} \over {n}}} \cdot e ^ {\pi i} = e ^ {k \cdot {{2 \pi i} \over {n}}} \cdot (-1) = -e ^ {k \cdot {{2 \pi i} \over {n}}} = -{{\omega}_{n, k}}$$
$${\omega^2_{n, {k + {n \over 2}}}} = (-{{\omega}_{n, k}}) ^ 2 = {\omega^2_{n, k}} =  {{\omega}_{{n \over 2}, k}}$$

Пусть $k \in \{ 0, 1, 2, \dots {n \over 2} - 1 \}$, тогда, используя первое свойство, получим:
$$ y_k = A(\omega_{n, k}) = A_0(\omega ^ 2_{n, k}) + \omega_{n, k} \cdot A_1(\omega ^ 2_{n, k}) = A_0({{\omega}_{{n \over 2}, k}}) + \omega_{n, k} \cdot A_1({{\omega}_{{n \over 2}, k}}) $$
Степени многочленов $A_0(x)$ и $A_1(x)$ равны ${n \over 2} - 1$, значит $y^0_k = A_0({{\omega}_{{n \over 2}, k}})$ и $y^1_k = A_1({{\omega}_{{n \over 2}, k}})$. Подставим в выражение выше, получим:
$$ y_k = y^0_k + \omega_{n, k} \cdot y^1_k $$
Таким образом, мы вычислили половину значений $DFT(A)$, используя значения $DFT(A_0)$ и $DFT(A_1)$.

Используем второе и третье свойства:
$$ y_{k + {n \over 2}} = A(\omega_{n, {k + {n \over 2}}}) = A_0(\omega ^ 2_{n, {k + {n \over 2}}}) + \omega_{n, {k + {n \over 2}}} \cdot A_1(\omega ^ 2_{n, {k + {n \over 2}}}) = A_0({{\omega}_{{n \over 2}, k}}) - \omega_{n, k} \cdot A_1({{\omega}_{{n \over 2}, k}}) $$
Подставим $y^0_k$ и $y^1_k$ в выражение:
$$ y_{k + {n \over 2}} = y^0_k - \omega_{n, k} \cdot y^1_k $$

Итак, мы получили $DFT(A)$ с помощью значений $DFT(A_0)$ и $DFT(A_1)$. То есть мы разбили исходную задачу на две поздачи вдвое меньшего размера и, решив их, получили решение исходной задачи. Такая схема называется <<разделяй-и-властвуй>> и имеет известную вычислительную сложность $O(n \cdot \log(n))$.

Всё вышеописанное называется быстрым преобразование Фурье.

\pagebreak

\subsection*{Обратное быстрое преобразование Фурье}
Выше дано определение ДПФ в матричном виде. БПФ является методом быстрого вычисления ДПФ. Обратное преобразование Фурье строится так же перемножением матрицы на вектор. Для быстрого вычисления обратного преобразования Фурье достаточно обозначить $\omega_{n, k} = {\omega}_{n} ^ {-k} = e ^ {-k \cdot {{2 \pi i} \over {n}}}$ в формулах для БПФ, получаться аналогичные формулы. На последнем шаге вычисления обратного преобразования Фурье нужно не забыть разделить все полученные коэффициенты на $n$.

\subsubsection*{Листинг рекурсивного БПФ}
Ниже приведена рекурсивная и простая реализация БПФ. Функция принимает вектор комплексных чисел и записывает в него же результат вычисления прямого или обратного БПФ.
\begin{lstlisting}[language=C++]
using complex = std::complex<double>;
using vc = std::vector<complex>;

const double PI = std::acos(-1);

void fft(vc & a, bool invert) {
    size_t n = a.size();
    if (n == 1) {
        return;
    }
    vc a0(n / 2), a1(n / 2);
    for (size_t i = 0, j = 0; j < n; ++i, j += 2) {
        a0[i] = a[j];
        a1[i] = a[j + 1];
    }
    fft(a0, invert);
    fft(a1, invert);
    double phi = 2.0 * PI / n;
    if (invert) {
        phi = -phi;
    }
    complex w(1), wn(std::cos(phi), std::sin(phi));
    for (size_t i = 0; i < n / 2; ++i) {
        a[i] = a0[i] + w * a1[i];
        a[n / 2 + i] = a0[i] - w * a1[i];
        if (invert) {
            a[i] /= 2.0;
            a[n / 2 + i] /= 2.0;
        }
        w *= wn;
    }
}
\end{lstlisting}

\pagebreak

\subsubsection*{Ускорение алгоритма}
Заметим, что в приведённой реализации используется $O(n \cdot \log(n))$ памяти, из-за чего константа в асимптотике алгоритма довольно велика. Можно отказаться от рекурсивного выделения памяти, если переупорядочить элементы в исходном векторе.

Рассмотрим самый нижний уровень рекурсии и то, какие коэффициенты исходного многочлена будут использоваться при вычислении.

Изначальный вектор (окружен фигурными скобками)
$$ \{ a_0, a_1, a_2, a_3, a_4, a_5, a_6, a_7 \} $$
распадается на два вызова (окружены квадратными скобками)
$$\{ [ a_0, a_2, a_4, a_6 ], [ a_1, a_3, a_5, a_7 ] \}$$

На следующем уровне рекурсии
$$\{ [ a_0, a_2, a_4, a_6 ], [ a_1, a_3, a_5, a_7 ] \}$$
распадается на четыре вызова (окружены круглыми скобками)
$$\{ [ (a_0, a_4), (a_2, a_6) ], [(a_1, a_5), (a_3, a_7)] \}$$

Полученный порядок называется поразрядно обратной перестановкой. Если мы посмотрим на какой позиции стоит тот или иной элемент, окажется, что если мы выполним реверс бит в двоичной записи изначальных позиций, то получим позиции на последнем уровне рекурсии. Например, $6_{10} = {110}_{2}$, выполним реверс бит, получим ${011}_{2} = {3}_{10}$, что соответствует индексы элемента $a_6$ на последнем уровне рекурсии.

Упорядочим элементы вектора $a$ как на последнем слою. Вычислим значения ДПФ для пар, окруженных круглыми скобками и запишем их вместо самих элементов, получим:
$$\{ [ (y^0_0, y^0_1), (y^1_0, y^1_0) ], [(a_1, a_5), (a_3, a_7)] \}$$
Теперь запишем новые значение ДПФ на позиции самих элементов:
$$\{ [ (y^0_0 + {\omega_{4, 0} } \cdot y^1_0, y^0_1 + {\omega_{4, 1} } \cdot y^1_1), (y^0_0 - {\omega_{4, 0} } \cdot y^1_0, y^0_1 - {\omega_{4, 1} } \cdot y^1_1) ], [(a_1, a_5), (a_3, a_7)] \}$$
Проделаем ту же операцию справа, получим:
$$ \{ [ y^0_0, y^0_1, y^0_2, y^0_3 ], [ y^1_0, y^1_1, y^1_2, y^1_3 ] \} $$
Видно, что элементы снова стоят подходящим образом, можно записывать значения ДПФ на позиции самих элементов:
$$ \{ [ y^0_0 + {\omega_{8, 0} } \cdot y^1_0, y^0_1, y^0_2, y^0_3 ], [ y^0_0 - {\omega_{8, 0} } \cdot y^1_0, y^1_1, y^1_2, y^1_3 ] \} $$
После всех вычислений получим:
$$ \{ y_0, y_1, y_2, y_3, y_4, y_5, y_6, y_7 \} $$
Таким образом мы смогли вычислить значения ДПФ без использования дополнительной памяти. Такой подход уменьшает константу в асимптотике. Временная сложность по-прежнему $O(n \cdot \log(n))$, а вот пространственная $O(1)$.

\subsubsection*{Листинг нерекурсивного БПФ}
Ниже приведена нерекурсивная реализация БПФ.
\begin{lstlisting}[language=C++]
using complex = std::complex<double>;
using vc = std::vector<complex>;

const double PI = std::acos(-1);

int rev_bits(int x, int lg_n) {
    int y = 0;
    for (int i = 0; i < lg_n; ++i) {
        y = y << 1;
        y ^= (x & 1);
        x = x >> 1;
    }
    return y;
}

void fft(vc & a, bool invert) {
    int n = a.size();
    int lg_n = 0;
    while ((1 << lg_n) < n) {
        ++lg_n;
    }
    for (int i = 0; i < n; ++i) {
        if (i < rev_bits(i, lg_n)) {
            swap(a[i], a[rev_bits(i, lg_n)]);
        }
    }
    for (int layer = 1; layer <= lg_n; ++layer) {
        int cluster = 1 << layer;
        double phi = (2.0 * PI) / cluster;
        if (invert) {
            phi *= -1;
        }
        complex wn = complex(std::cos(phi), std::sin(phi));
        for (int i = 0; i < n; i += cluster) {
            complex w(1);
            for (int j = 0; j < cluster / 2; ++j) {
                complex u = a[i + j];
                complex v = a[i + j + cluster / 2] * w;
                a[i + j] = u + v;
                a[i + j + cluster / 2] = u - v;
                w *= wn;
            }
        }
    }
    if (invert) {
        for (int i = 0; i < n; ++i) {
            a[i] /= n;
        }
    }
}
\end{lstlisting}

\subsubsection*{Перемножение двух чисел с использованием БПФ}
Пусть $a = 7310$ и $b = 2468$. Можно записать $a = 7 \cdot 10 ^ 3 + 3 \cdot 10 ^ 2 + 1 \cdot 10 + 0$ и $b = 2 \cdot 10 ^ 3 + 4 \cdot 10 ^ 2 + 6 \cdot 10 + 8$.

Пусть $A(x) = 0 + 1 \cdot x + 3 \cdot x ^ 2 + 7 \cdot x ^ 3$ и $B(x) = 8 + 6 \cdot x + 4 \cdot x ^ 2 + 2 \cdot x ^ 3$. Тогда $A(10) = a$ и $B(10) = b$. Если мы вычислим ДПФ для многочленов $A(x)$ и $B(x)$, то перемножив их значения ДПФ в одних и тех же точках, мы получим значения ДПФ для $(A \cdot B) (x)$, то есть
$$DFT(A \cdot B) = DFT(A) \cdot DFT(B)$$

Применив обратное преобразование Фурье к полученному и вектору, мы получим коэффициенты многочлена $(A \cdot B) (x)$. Так мы смогли перемножить два многочлена, используя БПФ:
$$ (A \cdot B) (x) = InverseDFT(DFT(A \cdot B)) = InverseDFT(DFT(A) \cdot DFT(B)) $$

Так как $A$ и $B$ изначально представляли числа $a$ и $b$, выполним перенос разрядов, чтобы получить многочлен, предсталвяющий $a \cdot b$.

Такой подход позволяет перемножать два многочлена за $O(n \cdot \log(n))$. Мы смогли представить два числа в виде многочленов, таким образом смогли перемножить два числа за такую же асимптотику $O(n \cdot \log(n))$.

\pagebreak
